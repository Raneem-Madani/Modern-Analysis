\documentclass[12pt]{article}
\usepackage[utf8]{inputenc}
\title{\Huge{Modern-Analysis 2\\Lecture-22}}
\usepackage{natbib}
\usepackage{graphicx}
\usepackage{amsmath}
\usepackage{amssymb}
\usepackage{mathtools}
\usepackage{xcolor}
\renewcommand{\baselinestretch}{1.5}
\newtheorem{theorem}{Theorem}
\usepackage{pgf,tikz}
\usepackage{mathrsfs}
\usetikzlibrary{arrows,decorations.markings}
\usepackage{graphicx}
\usepackage{tikz,lipsum,lmodern}
\usepackage[a4paper, total={6in,8in}]{geometry}
\usepackage[most]{tcolorbox}
\begin{document}
\maketitle
\newpage


\section*{55:Riemann-Stieltjes Integration with Respect to Functions of Bounded Variation}
\begin{tcolorbox}[colback=blue!5!white,colframe=blue!75!black,title=Theorem ]
     $\alpha$ continuous and differentiate on $[a,b]$, if $f,\alpha'\in\mathscr{R}[a,b]\Rightarrow f\in\mathscr{R}_\alpha[a,b]$ and
$$ \tcbhighmath{\int_{a}^{b} f(x)d\alpha(x)=\int_{a}^{b}f(x)\alpha'(x)dx}$$

\end{tcolorbox}

{{\color{blue}\underline {Example}} Find:}
\begin{enumerate}
    \item  $\int_0^1 x^2 d x^2=\int_0^2 x^2(2x)dx=\frac{1}{2}x^4|_0^2=\frac{16}{2}=8$
    \item $\int_0^2 [x]dx^2 =\int_0^1 [x]dx^2+\int_1^2 [x]dx^2=\int_0^1 0dx^2+\int_1^2 1dx^2=0+\int_1^2 2xdx=3$
    \\
\end{enumerate}


\begin{tcolorbox}[colback=blue!5!white,colframe=blue!75!black,title=Corollary "Fundamental theorem"]
   Let $f$ be continuous and differentiable on $[a,b]$. If $f'\in\mathscr{R}[a,b]\Longrightarrow$ 
$$ \tcbhighmath{\int_a^b f'(x)=f(b)-f(a)}$$
{\color{blue}\underline {Proof:}} $\int_a^b f'(x)dx=\int_a^b 1 f'(x) dx=\int_a^b1 df(x)=f(b)-f(a)$
\\
\end{tcolorbox}

{\color{blue}\underline {Example:}} Find $\int_1^4 \sqrt{x^2+1}d(x^2+3)=\int_1^4\sqrt{x^2+1}(2x)dx=\frac{2}{3}(17^{\frac{3}{2}}-1)$ \newpage

\section*{Chapter X: Sequences and Series of Functions}
\section*{\large{sec-60:Pointwise Convergence and Uniform Convergence}}
$$\{f_n\}_{n=1}^\infty,f_n:x\mapsto\mathbb{R},d(x,y)=|x-y|$$

\begin{tcolorbox}[colback=blue!5!white,colframe=blue!75!black,title=Definition:]
   $\{f_n\}$, $f_n:x\longmapsto\mathbb{R}$, $f:x\longmapsto\mathbb{R}$, We say that $\{f_n\}_{n=1}^\infty$ converges pointwise to $f$ on $X$ if $\lim_{n\mapsto\infty}f_n(x)=f(x),\forall x\in X.$
\end{tcolorbox}

{\color{blue}\underline {Example:}} $f_n (x)=x^2, 0\leq x\leq 1$\\
$x=\frac{1}{2},\frac{1}{4},...$\\
$f(\frac{1}{2}),f(\frac{1}{4}),...\mapsto 0$\\
$f(1)\mapsto 1$\\
so $f_n$ converges pointwise to $f$, where 
$f(x)=\begin{cases} 
      0 &,0\leq x< 1\\
      1 &,x=1
\end{cases}$



\begin{tcolorbox}[colback=blue!5!white,colframe=blue!75!black,title=Definition:]
   $\{f_n\}$, $f_n:x\longmapsto\mathbb{R}$, $f:x\longmapsto\mathbb{R}$, We say that $\{f_n\}_{n=1}^\infty$ converges uniformly to $f$ on $X$ if $\epsilon >0 , \exists $N $\in N$ s.t $|f_n(x) - f(x) |< \epsilon $\\ $ \forall n \geq N ,  \forall x \in X.$  
\end{tcolorbox}

\begin{tcolorbox}[colback=blue!5!white,colframe=blue!75!black,title=Theorem ]
     Uniform Converges $\Rightarrow$ pointwise convergence . 
  $$\tcbhighmath{Notation : 
      f_n \rightarrow f {\color{red}{(point-wise)}} \hspace{2mm}
  ,  f_n \rightrightarrows  f
     {\color{red}{(uniform)}}}$$
\end{tcolorbox}

\begin{tcolorbox}[colback=blue!5!white,colframe=blue!75!black,title=Theorem:]
   if $\{f_n\}$, $f_n:M \longmapsto\mathbb{R}$, $f:M\longmapsto\mathbb{R}$, We say that $\{f_n\}_{n=1}^\infty${\color{red}{ converges uniformly}} to $f$ on $M$ if $f_n$ is cont. at $a$ Then $f$ is cont. at $a$ \\ 
   {\color{blue}\underline {PF: }} Let $d$ be the metric of $M$ \\ $f_n \Rightarrow f$ \\ $f_n$ is cont. at $a \Rightarrow f$ is cont. at $a$ \\
   So $\forall \epsilon >0  \exists$ N s.t $|f_n(x)-f(x)| < \frac{\epsilon}{3} ,$ $\forall x \in M$ \\ 
   $\forall \epsilon >0 , |f(x)-f(a)| = |f(x)-f_N(X)+f_N(X)-f_N(a)+f_N(a) -f(a)|$ \\
   $\leq |f(x)-f_N(x)| +|f_N(x)-f_N(a)|+|f_N(a)-f(a)|$\\
   $ < \frac{\epsilon}{3} + \frac{\epsilon}{3} + \frac{\epsilon}{3} =\epsilon$
   \hspace{6mm}{\color{red}{if\hspace{1mm} $d(x,a) < \delta$}} \\
   Thus $f$ is cont. at $a$ 
\end{tcolorbox}
\begin{tcolorbox}[colback=blue!5!white,colframe=red!75!black,title=Note :]
* Uniform $\Rightarrow $ Pointwise  but The converse is false . \\ Counter example : $f_n$ {\color{red}{converges pointwise}} to $f$, where 
$f(x)=\begin{cases} 
      0 &,0\leq x< 1\\
      1 &,x=1
\end{cases}$\\
but if you take the limit it's not cont. so it.s{\color{red}{ not convergence uniformly}}
\end{tcolorbox}
\end{document}