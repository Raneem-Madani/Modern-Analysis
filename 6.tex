\documentclass{article}
\usepackage[utf8]{inputenc}
\usepackage{natbib}
\usepackage{graphicx}
\usepackage{amsmath}
\usepackage{amssymb}
\usepackage{mathtools}
\usepackage{xcolor}
\renewcommand{\baselinestretch}{1.5}
\newtheorem{theorem}{Theorem}
\usepackage{pgf,tikz}
\usepackage{mathrsfs}
\usetikzlibrary{arrows,decorations.markings}
\usepackage{graphicx}
\title{\Huge Homework fourth Week}
\author{\huge Raneem Madani\\ Student Number: 11820975 }
\date{\today}
\usepackage{tikz,lipsum,lmodern}
\usepackage[most]{tcolorbox}
\usepackage{amsmath}
\begin{document}
\maketitle
%%%%%%%%%%%%%%%%%%%%%%%%%%%%%%%%%%%%%%%%%%%%%%%%%%%%%%%%%%%%%%%%%%5%%%
\section*{\huge{IX.The Riemann-Stieltjes Integral}}
\section*{51.Riemann-Stieltjes Integration with Respect to an
Increasing Integrator}
%$$$$$$$$$$$$$$$$$$$$$$$$$$$$$$$$4$$$$$$$$4
\begin{tcolorbox}[enhanced,attach boxed title to top center={yshift=-3mm,yshifttext=-1mm},
  colback=blue!5!white,colframe=blue!75!black,colbacktitle=red!80!black,
  title=Exercise 5:.,fonttitle=\bfseries,
  boxed title style={size=small,colframe=red!50!black} ]
  \textit{\underline {\color{blue}Solution:}}
$f$ is bounded function on $[a,b]$ and $\alpha$ increasing on $[a,b]$. We want to show that:
$$\overline{\int}_a^b f d\alpha=-\underline{\int}_a^b (-f)d\alpha$$
Let $p=\{x_0,x_1...x_n\}$ be any partion of $[a,b]$.\\
We know that $\overline{\int}^a_b f d\alpha=inf U(f,p)$, so $\overline{\int}^a_b f d\alpha \leq U(f,p)$\\
And $\underline{\int}_a^b (-f)d\alpha=sup L(-f,p)$, so $\underline{\int}_a^b (-f)d\alpha\geq L(-f,p)$\\
\noindent\rule{8cm}{0.4pt}\\
{\color{red}Claim-1:} $\overline{\int}_a^b f d\alpha\geq -\underline{\int}_a^b (-f)d\alpha$\\
{\color{red}Proof the claim:} we know that $\left( \overline{\int}_a^b f d\alpha\geq\underline{\int}_a^b fd\alpha\right)\times -1\Rightarrow$\\
$\left( -\overline{\int}_a^b f d\alpha\leq-\underline{\int}_a^b fd\alpha=\underline{\int}_a^b -fd\alpha\right)\times -1\Rightarrow$\\
 $\overline{\int}_a^b f d\alpha\geq-\underline{\int}_a^b -fd\alpha$\\
 \noindent\rule{8cm}{0.4pt}\\
 {\color{red}Claim-2:} $\overline{\int}_a^b f d\alpha\leq -\underline{\int}_a^b (-f)d\alpha$\\
{\color{red}Proof the claim:} we know that $\left( \overline{\int}_a^b -f d\alpha\geq\underline{\int}_a^b -fd\alpha\right)\times -1\Rightarrow$\\
 $\overline{\int}_a^b f d\alpha =\overline{\int}_a^b -(-f) d\alpha=-\overline{\int}_a^b -f d\alpha\leq-\underline{\int}_a^b -fd\alpha\Rightarrow$\\
  $\overline{\int}_a^b f d\alpha\leq -\underline{\int}_a^b (-f)d\alpha$\\
 \noindent\rule{8cm}{0.4pt}\\
 so from claim 1 and claim 2 we have $\overline{\int}_a^b f d\alpha=-\underline{\int}_a^b (-f)d\alpha$, so we are done.
\end{tcolorbox}

%&&&&&&&&&&&&&&&&&&&&&&&&&&&&&&&&&&
\begin{tcolorbox}[enhanced,attach boxed title to top center={yshift=-3mm,yshifttext=-1mm},
  colback=blue!5!white,colframe=blue!75!black,colbacktitle=red!80!black,
  title=Exercise 12:.,fonttitle=\bfseries,
  boxed title style={size=small,colframe=red!50!black} ]
  \textit{\underline {\color{blue}Solution:}}
$(a)$-Let $p=\{x_0,x_1...x_n\}$ be any partion of $[0,2]$.\\
$U(f,p)=\sum_{k=1}^n M_k\Delta\alpha_k$, and $M_k=1$, so $\sum_{k=1}^n\Delta\alpha_k$\\
    $=x_1-x_0+x_2-x_1+...+x_n-x_{n-1}=x_n-x_0=1-0=1$\\
    and 
    $L(f,p)=\sum_{k=1}^n m_k\Delta\alpha_k$, and $m_k=0$,\ so
    $0\sum_{k=1}^n\Delta\alpha_k=0$, since $U(f,p)\neq L(f,P)\Rightarrow f\notin\mathscr{R}_\alpha[0,2]$
    \\
  \noindent\rule{8cm}{0.4pt}\\
    $(b)$-Let $s=\{x_0,x_1...x_n\}$ be any partion of $[0,2]$.\\
    $U(g,s)=\sum_{k=1}^n M_k\Delta\alpha_k$, and $M_k=1$, so $\sum_{k=1}^n\Delta\alpha_k$\\
    $=x_1-x_0+x_2-x_1+...+x_n-x_{n-1}=x_n-x_0=1-0=1$\\
    and $L(g,s)=\sum_{k=1}^n m_k\Delta\alpha_k$, and $m_k=0$, so
    $0\sum_{k=1}^n\Delta\alpha_k=0$
    since $U(g,s)\neq L(g,s)\Rightarrow g\notin\mathscr{R}_\alpha[0,2]$\\
     \noindent\rule{8cm}{0.4pt}\\
     %&&&&&&&&&&&&&&&&&&&&&&&&&&&&&&&&&&&&&
$(c)$-Let $p=\{x_0,x_1...x_n\}$ be any partion of $[0,2]$, $\alpha(x)=x$.\\
$U(f,p)=\sum_{k=1}^n M_k\Delta\alpha_k$, and $M_k=1$, so $\sum_{k=1}^n\Delta\alpha_k$\\
    $=x_1-x_0+x_2-x_1+...+x_n-x_{n-1}=x_n-x_0=2-0=2$\\
    and $L(f,p)=\sum_{k=1}^n m_k\Delta\alpha_k$, and $m_k=0$,\ so
    $0\sum_{k=1}^n\Delta\alpha_k=0$, since $U(f,p)\neq L(f,p)\Rightarrow f\notin\mathscr{R}_\alpha[0,2]$\\
    
    Let $s=\{x_0,x_1...x_n\}$ be any partion of $[0,2]$, $\alpha(x)=x$.\\
$U(g,s)=\sum_{k=1}^n M_k\Delta\alpha_k$, and $M_k=1$, so $\sum_{k=1}^n\Delta\alpha_k$\\
    $=x_1-x_0+x_2-x_1+...+x_n-x_{n-1}=x_n-x_0=2-0=2$\\
    and $L(g,s)=\sum_{k=1}^n m_k\Delta\alpha_k$, and $m_k=0$,\ so
    $0\sum_{k=1}^n\Delta\alpha_k=0$, since $U(g,s)\neq L(g,s)\Rightarrow g\notin\mathscr{R}_\alpha[0,2]$\\
    
\end{tcolorbox}

%$$$$$$$$$$$$$$$$$$$$$$$$$$$$$$$$$$$$$$
\begin{tcolorbox}[enhanced,attach boxed title to top center={yshift=-3mm,yshifttext=-1mm},  colback=blue!5!white,colframe=blue!75!black,colbacktitle=red!80!black,
  title=Exercise 18:.,fonttitle=\bfseries,
  boxed title style={size=small,colframe=red!50!black} ]
  \textit{\underline {\color{blue}Solution:}}
\begin{itemize}
    \item Let $\alpha(x)=x$ and $f(x)=$
    $\begin{cases}
     -1,& x\in\mathbb{Q}\cap[0,1]\\
      1,& x\in\mathbb{Q}^c\cap[0,1]
    \end{cases}$\\
    Now $\overline{\int}_a^b f dx=\sum_{k=1}^n M_k\Delta\alpha_k$, and $M_k=1$, so $\sum_{k=1}^n\Delta\alpha_k$\\
    $=x_1-x_0+x_2-x_1+...+x_n-x_{n-1}=x_n-x_0=1-0=1$\\
    and $\underline{\int}_a^b=\sum_{k=1}^n m_k\Delta\alpha_k$, and $m_k=1$, so
    $-1\sum_{k=1}^n\Delta\alpha_k=$\\
    $-1[x_1-x_0+x_2-x_1+...+x_n-x_{n-1}]=-1[x_n-x_0]=-1[1-0]=-1$\\
    so $\overline{\int}_a^b=1\leq -1=\underline{\int}_a^b$ so $f\notin\mathscr{R}_\alpha[0,1]$
    \item But $|f(x)|=$
     $\begin{cases}
      1, & x\in\mathbb{Q}\cap[0,1]\\
      1,& x\in\mathbb{Q}^c\cap[0,1]
    \end{cases}$\\
    so $|f(x)|:=\{1, x\in[0,1]$\}, and $\overline{\int}_a^b=0=\underline{\int}_a^b$, so $|f|\in\mathscr{R}_\alpha[0,1]$
\end{itemize} 
\begin{tcolorbox}[colback=blue!5!white,colframe=blue!75!black]
\begin{center}
     So $|f|\in\mathscr{R}_\alpha[a,b]$, but $f\notin\mathscr{R}_\alpha[a,b]$
\end{center}
 
\end{tcolorbox}
\end{tcolorbox}

%$$$$$$$$$$$$$$$$$$$$$$$$$$$$$$$$$$$$$$%%%%%%%%%%%%%%%%%%%%%%%%%%%%%%%%%%%%%%%%%%%%%%%%%%%%%%
\section*{54.Functions of Bounded Variation}

%$$$$$$$$$$$$$$$$$$$$$$$$$$$$$$$$$$$$$$$$$$$$$
\begin{tcolorbox}[enhanced,attach boxed title to top center={yshift=-3mm,yshifttext=-1mm},
  colback=blue!5!white,colframe=blue!75!black,colbacktitle=red!80!black,
  title=Exercise 5:.,fonttitle=\bfseries,
  boxed title style={size=small,colframe=red!50!black} ]
  \textit{\underline {\color{blue}Solution:}}
Since $\alpha(x)\in BV[a,b]$ so $\sum_{k=1}^n|\alpha(x_k)-\alpha{(x_{k-1}}|\leq c,c\in\mathbb{R}$, and we have $|\alpha(x)|>M$, so $\frac{1}{\alpha(x)}\leq \frac{1}{M}$.\\
Now  $\sum_{k=1}^n|\frac{1}{\alpha(x_k)}-\frac{1}{\alpha(x_{k-1})}|= \sum_{k=1}^n|\frac{\alpha(x_{k-1})-\alpha(x_k)}{\alpha(x_{k-1})\alpha(x_k)}|\leq\frac{c}{M^2}$.\\
so $\sum_{k=1}^n|\frac{1}{\alpha(x_k)}-\frac{1}{\alpha{(x_{k-1})}}|$ bounded.
\end{tcolorbox}

\end{document}
